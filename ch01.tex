\blogchapter{chap:PythonPrimer}{Python Primer}
\blogsection{sec:Exercises}{Exercises}{}{}
\label{sec:Exercises}

I doubted doing the exercises for the first chapter for like an hour while I walked to my home from work. Hell, I even doubted doing the whole book, Data Structures? But even though I know most of the theory, I can't extract it like its hot. The other day I was implementing a solution to a VERY simple problem and I had trouble recalling the data structure with constant time complexity removal (answer: linked list). After that I felt extremely bad for myself, if I can't use data structures as if they were cards in a poker match then I am not really a Software Developer. My job is to juggle multiple solutions to craft one that is at least perfect.

\blogsubsection{ssec:reshaping}{Reshaping the future}{}{} \label{subsec:reshaping}

It feels like I've been doing it wrong during the last years, I've been focusing on developing a complete top notch quality solution, worrying about code quality, unit testing, commenting my code, learning back end and front end, and learning to make my own architectures, hell, my last subject was starting to learn QT.

It ends now, I am focusing full force on Computer Science, and just a tiny bit of time learning new technologies. It is amazing what I am seeing everywhere, Angular JS, Node, MVVM on WPF, QML, it is just fantastic. But trying to cover everything is exhausting, I feel like I am getting nowhere. I need to up the ante, and I feel the only way is to have the most solid foundation, which is Computer Science, plain and simple.

Which made it even harder to get the first chapter done, and even more the second. Since both chapters are Python specific. 

But let's be serious, they should be no problem at this height. 

\blogsubsubsection{sssec:ex1_1}{R-1.1}{}{} \label{subsubsec:ex1_1}

Write a short Python function, <code>is_multiple(n, m)</code>, that takes two integer values and returns <code>True</code> if $n$ is a multiple of $m$, that is, $n = mi$ for some integer $i$, and <code>False</code> otherwise.

\lstset{language=Python}
\begin{lstlisting}[label=ex1.1,caption=Exercise R-1.1]
def is_multiple(n, m):
    return True if (n \% m == 0) else False
\end{lstlisting}