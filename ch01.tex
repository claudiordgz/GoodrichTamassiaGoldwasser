\blogchapter{chap:Format}{Format}

All exercises will be presented with their own Python Doctest documentation to allow testing. To run them in your own python package you can copy paste the text and add a main like the following:

\begin{lstlisting}[title=Running Doctest]
if __name__ == "__main__":
    import doctest
    doctest.testmod()
\end{lstlisting}

This is just to try to keep it as simple as possible while adding how to run the code in your own work environment.

\ProTip{JetBrains Pycharm is awesome, I really recommend it, plus they got a Community Edition if you are pennyless like me. The colors, the functionality it just rocks. \textbf{Plus the IDE can run the examples without the need of using a main function.}}


\ProTip{I like to use Anaconda for my Python distro, but the standalone Python $2.7$ or $>=3$ works too.}


\blogchapter{chap:PythonPrimer}{Python Primer}

The first chapter in the book is all about learning to handle Python syntax. Subjects include objects, control flow, functions, I/O operations, exceptions, iterators and generators, namespaces, modules, and scope. There is nothing regarding python packaging to redistribute your own module, which is a subject of its own. 

\blogsubsection{ssec:Exercises}{Exercises}{}{}
\label{ssec:Exercises}

The exercises in the first chapter are fun, no joke. I've seen what's coming in chapter 2 and those exercises look terrible because they are open ended questions, but they are also important concepts.  

\blogsubsubsection{sssec:ex1_1}{R-1.1}{chap:PythonPrimer}{} \label{sssec:ex1_1}

Write a short Python function, \texttt{is\_multiple(n, m)}, that takes two integer values and returns \texttt{True} if $n$ is a multiple of $m$, that is, $n = mi$ for some integer $i$, and \texttt{False} otherwise.

\lstinputlisting[title=Exercise R-1.1]{../../GoodrichTamassiaGoldwasser/ch01/r11.py}

\blogsubsubsection{sssec:ex1_2}{R-1.2}{chap:PythonPrimer}{} \label{sssec:ex1_2}

Write a short Python function, \texttt{is\_even(k)}, that takes an integer value and returns \texttt{True} if $k$ is even, and \texttt{False} otherwise. However, your function cannot use the multiplication, modulo, or division operators.

\begin{lstlisting}[title=Exercise R-1.2]
def is_even(k):
    """Return True if n is even 
		else returns False

    >>> is_even(10)
    True
    >>> is_even(9)
    False
    >>> is_even(11)
    False
    >>> is_even(13)
    False
    >>> is_even(1025)
    False
    >>> is_even("test")
    Number must be Integer values
    """
    try:
        return int(k) & 1 == 0
    except ValueError:
        print("Numbers must be Integer values")
\end{lstlisting}
\blogsubsubsection{sssec:ex1_3}{R-1.3}{chap:PythonPrimer}{} \label{sssec:ex1_3}

Write a short Python function, \texttt{minmax(data)}, that takes a sequence of one or more numbers, and returns the smallest and largest numbers, in the form of a tuple of length two. Do not use the built-in functions \texttt{min} or \texttt{max} in implementing your solution.

\lstinputlisting[title=Exercise R-1.3]{../../GoodrichTamassiaGoldwasser/ch01/r13.py}

\blogsubsubsection{sssec:ex1_41_5}{R-1.4 \& R-1.5}{chap:PythonPrimer}{} \label{sssec:ex1_41_5}

Write a short Python function that takes a positive integer $n$ and returns the sum of the squares of all the positive integers smaller than $n$.

Give a single command that computes the sum from Exercise R-1.4, relying on Python's comprehension syntax and the built-in sum function.

\begin{lstlisting}[title=Exercise R-1.4 \& R-1.5]
def sum_of_squares(n):
    """Sum of squares of postive integers
    smaller than n

    Args:
        n (int): Highest number

    >>> sum_of_squares(10)
    285
    >>> sum_of_squares(20)
    2470
    >>> sum_of_squares(500)
    41541750
    >>> sum_of_squares(37)
    16206
    >>> sum_of_squares(-1)
    False
    """
    return sum([pow(x,2) for x in range(n)]) if n > 0 else False
\end{lstlisting}
\blogsubsubsection{sssec:ex1_61_7}{R-1.6 \& R-1.7}{chap:PythonPrimer}{} \label{sssec:ex1_61_7}

Write a short Python function that takes a positive integer $n$ and returns the sum of the squares of all the odd positive integers smaller than $n$.

Give a single command that computes the sum from Exercise R-1.6, relying on Python's comprehension syntax and the built-in sum function.

\begin{lstlisting}[title=Exercise R-1.6 \& R-1.7]
def sum_of_odd_squares(n):
    """Sum of squares of odd postive integers
    smaller than n

    Args:
        n (int): Highest number

    >>> sum_of_odd_squares(10)
    165
    >>> sum_of_odd_squares(20)
    1330
    >>> sum_of_odd_squares(500)
    20833250
    >>> sum_of_odd_squares(37)
    7770
    >>> sum_of_odd_squares(-1)
    False
    """
    return sum([pow(x,2) for x in range(1, n, 2)]) if n > 0 else False
\end{lstlisting}
\blogsubsubsection{sssec:ex1_8}{R-1.8}{chap:PythonPrimer}{} \label{sssec:ex1_8}

Python allows negative integers to be used as indices into a sequence, such as a string. If string $s$ has length $n$, and expression $s[k]$ is used for index $-n\leq k<0$, what is the equivalent index $j\geq 0$ such that $s[j]$ references the same element?

\begin{lstlisting}[title=Exercise R-1.8]
def return_element(data, k):
    """Tells you the equivalent negative index

    Args:
        data (list of int): Simple array
        k (int): index you want to know
        the equivalent negative index

    Returns:
        (val, index)
        val (object): element at position k
        index: negative index of that position
    Examples:
        Here are some examples!

    >>> l = [2,3,4,5,6,7,8,9,10,11,10,9,8,7,6,5,4,3,2,1]
    >>> return_element(l, 0)
    (2, -20)
    >>> return_element(l, 1)
    (3, -19)
    >>> return_element(l, 2)
    (4, -18)
    """
    idx = k-len(data)
    return data[idx], idx if data else False
\end{lstlisting}