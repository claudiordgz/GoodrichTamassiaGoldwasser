\blogchapter{chap:PythonPrimer}{Python Primer}

The first chapter in the book is all about learning to handle Python syntax. Subjects include objects, control flow, functions, I/O operations, exceptions, iterators and generators, namespaces, modules, and scope. There is nothing regarding python packaging to redistribute your own module, which is a subject of its own. 

\blogsection{sec:Format}{Format}{}{}\label{sec:Format}

All exercises will be presented with their own Python Doctest documentation to allow testing. To run them in your own python package you can copy paste the text and add a main like the following:

\begin{lstlisting}[title=DoctestMain]
if __name__ == "__main__":
    import doctest
    doctest.testmod()
\end{lstlisting}

This is just to try to keep it as simple as possible while adding how to run the code in your own work environment.

\ProTip{JetBrains Pycharm is awesome, I really recommend it, plus they got a Community Edition if you are pennyless like me. The colors, the functionality it just rocks. \textbf{Plus the IDE can run the examples without the need of using a main function.}}


\ProTip{I like to use Anaconda for my Python distro, but the standalone Python $2.7$ or $>=3$ works too.}

\blogsubsection{ssec:Exercises}{Exercises}{}{}
\label{ssec:Exercises}

The exercises in the first chapter are fun, no joke. I've seen what's coming in chapter 2 and those exercises look terrible because they are open ended questions, but they are also important concepts.  
\newpage
\blogsubsubsection{sssec:ex1_1}{R-1.1}{chap:PythonPrimer}{} \label{sssec:ex1_1}

Write a short Python function, \texttt{is\_multiple(n, m)}, that takes two integer values and returns \texttt{True} if $n$ is a multiple of $m$, that is, $n = mi$ for some integer $i$, and \texttt{False} otherwise.

\lstinputlisting[title=Exercise R-1.1]{../../GoodrichTamassiaGoldwasser/ch01/r11.py}

\pagebreak
\blogsubsubsection{sssec:ex1_2}{R-1.2}{chap:PythonPrimer}{} \label{sssec:ex1_2}

Write a short Python function, \texttt{is\_even(k)}, that takes an integer value and returns \texttt{True} if $k$ is even, and \texttt{False} otherwise. However, your function cannot use the multiplication, modulo, or division operators.

\lstinputlisting[title=Exercise R-1.2]{../../GoodrichTamassiaGoldwasser/ch01/r12.py}

\pagebreak
\blogsubsubsection{sssec:ex1_3}{R-1.3}{chap:PythonPrimer}{} \label{sssec:ex1_3}

Write a short Python function, \texttt{minmax(data)}, that takes a sequence of one or more numbers, and returns the smallest and largest numbers, in the form of a tuple of length two. Do not use the built-in functions \texttt{min} or \texttt{max} in implementing your solution.

\lstinputlisting[title=Exercise R-1.3]{../../GoodrichTamassiaGoldwasser/ch01/r13.py}

\pagebreak
\blogsubsubsection{sssec:ex1_41_5}{R-1.4 \& R-1.5}{chap:PythonPrimer}{} \label{sssec:ex1_41_5}

Write a short Python function that takes a positive integer $n$ and returns the sum of the squares of all the positive integers smaller than $n$.

Give a single command that computes the sum from Exercise R-1.4, relying on Python's comprehension syntax and the built-in sum function.

\begin{lstlisting}[title=Exercise R-1.4 \& R-1.5]
def sum_of_squares(n):
    """Sum of squares of postive integers
    smaller than n

    Args:
        n (int): Highest number

    >>> sum_of_squares(10)
    285
    >>> sum_of_squares(20)
    2470
    >>> sum_of_squares(500)
    41541750
    >>> sum_of_squares(37)
    16206
    >>> sum_of_squares(-1)
    False
    """
    return sum([pow(x,2) for x in range(n)]) if n > 0 else False
\end{lstlisting}
\pagebreak
\blogsubsubsection{sssec:ex1_61_7}{R-1.6 \& R-1.7}{chap:PythonPrimer}{} \label{sssec:ex1_61_7}

Write a short Python function that takes a positive integer $n$ and returns the sum of the squares of all the odd positive integers smaller than $n$.

Give a single command that computes the sum from Exercise R-1.6, relying on Python's comprehension syntax and the built-in sum function.

\lstinputlisting[title=Exercise R-1.6 \& R-1.7]{../../GoodrichTamassiaGoldwasser/ch01/r16r17.py}
\pagebreak
\blogsubsubsection{sssec:ex1_8}{R-1.8}{chap:PythonPrimer}{} \label{sssec:ex1_8}

Python allows negative integers to be used as indices into a sequence, such as a string. If string $s$ has length $n$, and expression $s[k]$ is used for index $-n\leq k<0$, what is the equivalent index $j\geq 0$ such that $s[j]$ references the same element?

\lstinputlisting[title=Exercise R-1.8]{../../GoodrichTamassiaGoldwasser/ch01/r18.py}
\pagebreak
\blogsubsubsection{sssec:ex1_9}{R-1.9}{chap:PythonPrimer}{} \label{sssec:ex1_9}

What parameters should be sent to the range constructor, to produce a
range with values 50, 60, 70, 80?

\lstinputlisting[title=Exercise R-1.9]{../../GoodrichTamassiaGoldwasser/ch01/r19.py}
\pagebreak
\blogsubsubsection{sssec:ex1_10}{R-1.10}{chap:PythonPrimer}{} \label{sssec:ex1_10}

What parameters should be sent to the range constructor, to produce a
range with values 8, 6, 4, 2, 0, -2, -4, -6, -8?

\begin{lstlisting}[title=Exercise R-1.10]
def range_from_eigth():
    """ Return the list [8, 6, 4, 2, 0, -2, -4, -6, -8]
    :return:
        the list [8, 6, 4, 2, 0, -2, -4, -6, -8]
    >>> range_from_eigth()
    [8, 6, 4, 2, 0, -2, -4, -6, -8]
    """
    return range(8, -9, -2)
\end{lstlisting}
\pagebreak
\blogsubsubsection{sssec:ex1_11}{R-1.11}{chap:PythonPrimer}{} \label{sssec:ex1_11}

Demonstrate how to use Python's list comprehension syntax to produce the list [1, 2, 4, 8, 16, 32, 64, 128, 256].

\begin{lstlisting}[title=Exercise R-1.11]
def list_comprehension_example():
    """ Return list
    [1, 2, 4, 8, 16, 32, 64, 128, 256]

    :return:
        list: [1, 2, 4, 8, 16, 32, 64, 128, 256]

    >>> list_comprehension_example()
    [1, 2, 4, 8, 16, 32, 64, 128, 256]
    """
    return [pow(2,x) for x in range(9)]
\end{lstlisting}
\pagebreak
\blogsubsubsection{sssec:ex1_12}{R-1.12}{chap:PythonPrimer}{} \label{sssec:ex1_12}

\textbf{This is a random method, I usually stress test anything that is random since I am always uneasy about it. Because the test is so long I placed it separately.}

Python's random module includes a function choice(data) that returns a random element from a non-empty sequence. The random module includes a more basic function randrange, with parametrization similar to the built-in range function, that return a random choice from the given range. Using only the randrange function, implement your own version of the choice function.


\begin{lstlisting}[title=Exercise R-1.12]
def custom_choice(data):
		""" Returns random element from a non-empty sequence
		
		:returns:
				integer value of list
		"""
    import random
    return data[random.randrange(0,len(data))]
\end{lstlisting}

\ProTip{Stress test your random functions}

\textbf{Hereafter you can see the Doctest for this exercise. I separated it because it is hude.} In this case we are testing a random method that is part of a library, which is overkill. But the test is not really hard to make or design so it is not much of an investment in either time or money. 

\begin{lstlisting}[title=Exercise R-1.12 Doctest]
""" Python's random module includes a function choice(data) that returns a
random element from a non-empty sequence. The random module includes
a more basic function randrange, with parametrization similar to
the built-in range function, that return a random choice from the given
range. Using only the randrange function, implement your own version
of the choice function.

>>> data = [2,3,4,5,6,7,8,9,10,11,10,9,8,7,6,5,4,3,2,1]
>>> results = list()
>>> for x in range(len(data)*20):
...     val = custom_choice(data)
...     results.append(val in data)
>>> print(results)
[True, True, True, True, True, True, True, True, True, True, \
True, True, True, True, True, True, True, True, True, True, \
True, True, True, True, True, True, True, True, True, True, \
True, True, True, True, True, True, True, True, True, True, \
True, True, True, True, True, True, True, True, True, True, \
True, True, True, True, True, True, True, True, True, True, \
True, True, True, True, True, True, True, True, True, True, \
True, True, True, True, True, True, True, True, True, True, \
True, True, True, True, True, True, True, True, True, True, \
True, True, True, True, True, True, True, True, True, True, \
True, True, True, True, True, True, True, True, True, True, \
True, True, True, True, True, True, True, True, True, True, \
True, True, True, True, True, True, True, True, True, True, \
True, True, True, True, True, True, True, True, True, True, \
True, True, True, True, True, True, True, True, True, True, \
True, True, True, True, True, True, True, True, True, True, \
True, True, True, True, True, True, True, True, True, True, \
True, True, True, True, True, True, True, True, True, True, \
True, True, True, True, True, True, True, True, True, True, \
True, True, True, True, True, True, True, True, True, True, \
True, True, True, True, True, True, True, True, True, True, \
True, True, True, True, True, True, True, True, True, True, \
True, True, True, True, True, True, True, True, True, True, \
True, True, True, True, True, True, True, True, True, True, \
True, True, True, True, True, True, True, True, True, True, \
True, True, True, True, True, True, True, True, True, True, \
True, True, True, True, True, True, True, True, True, True, \
True, True, True, True, True, True, True, True, True, True, \
True, True, True, True, True, True, True, True, True, True, \
True, True, True, True, True, True, True, True, True, True, \
True, True, True, True, True, True, True, True, True, True, \
True, True, True, True, True, True, True, True, True, True, \
True, True, True, True, True, True, True, True, True, True, \
True, True, True, True, True, True, True, True, True, True, \
True, True, True, True, True, True, True, True, True, True, \
True, True, True, True, True, True, True, True, True, True, \
True, True, True, True, True, True, True, True, True, True, \
True, True, True, True, True, True, True, True, True, True, \
True, True, True, True, True, True, True, True, True, True, \
True, True, True, True, True, True, True, True, True, True]
"""
\end{lstlisting}

\pagebreak
\blogsubsubsection{sssec:ex1_13}{C-1.13}{chap:PythonPrimer}{} \label{sssec:ex1_13}

Write a pseudo-code description of a function that reverses a list of $n$ integers, so that the numbers are listed in the opposite order than they were before, and compare this method to an equivalent Python function for doing the same thing.

\lstinputlisting[title=Exercise C-1.13]{../../GoodrichTamassiaGoldwasser/ch01/c113.py}

\large\textbf{cProfile Results}

Here is a simple cProfile with the results. Time shows as $0.000$ but the number of function calls tell us our implementation is not that good. 

\texttt{25 function calls in 0.000 seconds}

\texttt{Ordered by: custom\_reverse}

\begin{tabular}{cccccl}
ncalls & tottime & percall & cumtime & percall & filename:lineno(function) \\
		$1$  &  $0.000$  &  $0.000$  &  $0.000$  &  $0.000$ &  $<$string$>$:1($<$module$>$) \\
		$1$  &  $0.000$  &  $0.000$  &  $0.000$  &  $0.000$ &  c113.py:14(custom\_reverse) \\
	 $21$  &  $0.000$  &  $0.000$  &  $0.000$  &  $0.000$ &  len \\
		$1$  &  $0.000$  &  $0.000$  &  $0.000$  &  $0.000$ & method 'disable' of '\_lsprof.Profiler' objects \\
		$1$  &  $0.000$  &  $0.000$  &  $0.000$  &  $0.000$ & range \\
		&&&&&\\
\end{tabular}

\texttt{3 function calls in 0.000 seconds}

\texttt{Ordered by: standard\_reverse}

\begin{tabular}{cccccl} 
ncalls & tottime & percall & cumtime & percall & filename:lineno(function) \\
		$1$  &  $0.000$  &  $0.000$  &  $0.000$  &  $0.000$  & $<$string$>$:1($<$module$>$) \\
		$1$  &  $0.000$  &  $0.000$  &  $0.000$  &  $0.000$  & c113.py:26(standard\_reverse) \\
		$1$  &  $0.000$  &  $0.000$  &  $0.000$  &  $0.000$  & method 'disable' of '\_lsprof.Profiler' objects \\ 
		&&&&&\\
\end{tabular}

\texttt{3 function calls in 0.000 seconds}

\texttt{Ordered by: other\_reverse}

\begin{tabular}{cccccl} 
ncalls & tottime & percall & cumtime & percall & filename:lineno(function) \\
		$1$  &  $0.000$  &  $0.000$  &  $0.000$  &  $0.000$  & $<$string$>$:1($<$module$>$) \\
		$1$  &  $0.000$  &  $0.000$  &  $0.000$  &  $0.000$  & c113.py:29(other\_reverse) \\
		$1$  &  $0.000$  &  $0.000$  &  $0.000$  &  $0.000$  & method 'disable' of '\_lsprof.Profiler' objects \\
		&&&&&\\
\end{tabular}



\pagebreak
\blogsubsubsection{sssec:ex1_14}{C-1.14}{chap:PythonPrimer}{} \label{sssec:ex1_14}

Write a short Python function that takes a sequence of integer values and determines if there is a distinct pair of numbers in the sequence whose product is odd.

In this method what we do first is remove all the repeated elements, then we just extract all the odd numbers. Multiply any of the odd numbers and you get an odd number. Any even number multiplied by an odd number returns an even number.

\lstinputlisting[title=Exercise C-1.14]{../../GoodrichTamassiaGoldwasser/ch01/c114.py}


\pagebreak
\blogsubsubsection{sssec:ex1_15}{C-1.15}{chap:PythonPrimer}{} \label{sssec:ex1_15}

Write a Python function that takes a sequence of numbers and determines if all the numbers are different from each other (that is, they are distinct).

We just build a set from the original list, if the length of the set is smaller, we have repeated numbers. Besides this, we could also sort the elements and look for a repeated element. 

\lstinputlisting[title=Exercise C-1.15]{../../GoodrichTamassiaGoldwasser/ch01/c115.py}


\pagebreak
\blogsubsubsection{sssec:ex1_161_17}{C-1.16 \& C-1.17}{chap:PythonPrimer}{} \label{sssec:ex1_161_17}

In our implementation of the scale function (page 25), the body of the loop executes the command \texttt{data[j] *= factor}. We have discussed that numeric types are immutable, and that use of the \texttt{*=} operator in this context causes the creation of a new instance (not the mutation of an existing instance).
How is it still possible, then, that our implementation of scale changes the actual parameter sent by the caller?

Had we implemented the scale function (page 25) as follows, does it work properly?

\begin{lstlisting}[title=The incorrect\_way]
def scale(data, factor):
	for val in data:
		val *= factor
\end{lstlisting}

Explain why or why not.

Well... here are some inmutable types in Python.

\begin{itemize}
	\item numerical types
	\item string types
	\item types
\end{itemize}

And here are some mutable types.

\begin{itemize}
	\item lists
	\item dicts
	\item classes
\end{itemize}

So taken by that premise, if we want a function that modifies numeric values in place, we would need a \texttt{class} that wraps that numeric value. This is not trivial. So we are going to take respectfully a class done by The Edwards Research group (\url{http://www.edwards-research.com/}), this class is presented here: \url{http://blog.edwards-research.com/2013/09/mutable-numeric-types-in-python/}.

It is everything you dreamed of, I give you, a mutable numeric. 

\lstinputlisting[title=Mutable Numeric]{../../GoodrichTamassiaGoldwasser/ch01/MutableNum.py}

Now onto the second question... if numeric types are not valid then of course that method is not going to work. You would need to do something \textbf{radical} like returning a \textbf{new list}. As follows:

\begin{lstlisting}[title=Returning new\_list]
def scale(data, factor):
    return [x*factor for x in data]
\end{lstlisting}

We can also change the list number by replacing the value in the list, remember that lists are mutable, so we can change that, we just can't change the number.

\lstinputlisting[title=Exercise C-1.16 \& C-1.17]{../../GoodrichTamassiaGoldwasser/ch01/c116c117.py}


\pagebreak
\blogsubsubsection{sssec:ex1_18}{C-1.18}{chap:PythonPrimer}{} \label{sssec:ex1_18}

Demonstrate how to use Python's list comprehension syntax to produce the list

\lstinputlisting[title=Exercise C-1.18]{../../GoodrichTamassiaGoldwasser/ch01/c118.py}


\pagebreak
\blogsubsubsection{sssec:ex1_19}{C-1.19}{chap:PythonPrimer}{} \label{sssec:ex1_19}

Demonstrate how to use Python's list comprehension syntax to produce the list [ a , b , c , ..., z ], but without having to type all 26 such characters literally.

\lstinputlisting[title=Exercise C-1.19]{../../GoodrichTamassiaGoldwasser/ch01/c119.py}

\pagebreak
\blogsubsubsection{sssec:ex1_20}{C-1.20}{chap:PythonPrimer}{} \label{sssec:ex1_20}

Python's random module includes a function \texttt{shuffle(data)} that accepts a list of elements and randomly reorders the elements so that each possible order occurs with equal probability. The random module includes a more basic function \texttt{randint(a, b)} that returns a uniformly random integer from a to b (including both endpoints). Using only the \texttt{randint} function, implement your own version of the shuffle function.

\lstinputlisting[title=Exercise C-1.20]{../../GoodrichTamassiaGoldwasser/ch01/c120.py}

\pagebreak
\blogsubsubsection{sssec:ex1_21}{C-1.21}{chap:PythonPrimer}{} \label{sssec:ex1_21}

Write a Python program that repeatedly reads lines from standard input until an \texttt{EOFError} is raised, and then outputs those lines in reverse order (a user can indicate end of input by typing ctrl-D).

\lstinputlisting[title=Exercise C-1.21]{../../GoodrichTamassiaGoldwasser/ch01/c121.py}

\pagebreak
\blogsubsubsection{sssec:ex1_22}{C-1.22}{chap:PythonPrimer}{} \label{sssec:ex1_22}

Write a short Python program that takes two arrays a and b of length n storing int values, and returns the dot product ofa and b. That is, it returns an array c of length n such that 
$c[i] = a[i] \cdot b[i]$, for $i = 0,...,n-1$.

\lstinputlisting[title=Exercise C-1.22]{../../GoodrichTamassiaGoldwasser/ch01/c122.py}
\pagebreak
\blogsubsubsection{sssec:ex1_23}{C-1.23}{chap:PythonPrimer}{} \label{sssec:ex1_23}

Give an example of a Python code fragment that attempts to write an element to a list based on an index that may be out of bounds. If that index is out of bounds, the program should catch the exception that results, and print the following error message:\\
``Don't try buffer overflow attacks in Python!''

\lstinputlisting[title=Exercise C-1.23]{../../GoodrichTamassiaGoldwasser/ch01/c123.py}
\pagebreak
\blogsubsubsection{sssec:ex1_24}{C-1.24}{chap:PythonPrimer}{} \label{sssec:ex1_24}

Write a short Python function that counts the number of vowels in a given character string.

\lstinputlisting[title=Exercise C-1.24]{../../GoodrichTamassiaGoldwasser/ch01/c124.py}
\pagebreak
\blogsubsubsection{sssec:ex1_25}{C-1.25}{chap:PythonPrimer}{} \label{sssec:ex1_25}

Write a short Python function that takes a strings, representing a sentence, and returns a copy of the string with all punctuation removed. For example, if given the string ``Let's try, Mike.'', this function would return:\\
``Lets try Mike''.

\lstinputlisting[title=Exercise C-1.25]{../../GoodrichTamassiaGoldwasser/ch01/c125.py}
\pagebreak
\blogsubsubsection{sssec:ex1_26}{C-1.26}{chap:PythonPrimer}{} \label{sssec:ex1_26}

Write a short program that takes as input three integers, $a$, $b$, and $c$, from the console and determines if they can be used in a correct arithmetic formula (in the given order), like "$a+b = c$," "$a = b-c$," or "$a \ast b = c$."

\lstinputlisting[title=Exercise C-1.26]{../../GoodrichTamassiaGoldwasser/ch01/c126.py}
\pagebreak
\blogsubsubsection{sssec:ex1_27}{C-1.27}{chap:PythonPrimer}{} \label{sssec:ex1_27}

In Section 1.8, we provided three different implementations of a generator that computes factors of a given integer. The third of those implementations, from page 41, was the most efficient, but we noted that it did not yield the factors in increasing order. Modify the generator so that it reports factors in increasing order, while maintaining its general performance advantages.

\lstinputlisting[title=Exercise C-1.27]{../../GoodrichTamassiaGoldwasser/ch01/c127.py}
\pagebreak
\blogsubsubsection{sssec:ex1_28}{C-1.28}{chap:PythonPrimer}{} \label{sssec:ex1_28}

The p-norm of a vector $v = (v_{1} ,v_{2} ,\ldots,v_{n} )$ in n-dimensional space is defined as:\\

\[
	\left\|v\right\| = \sqrt[p]{v^{p}_{1} + v^{p}_{2} + \ldots + v^{p}_{n}}
\]

For the special case of $p = 2$, this results in the traditional Euclidean norm, which represents the length of the vector. For example, the Euclidean norm of a two-dimensional vector with coordinates $(4,3)$ has a Euclidean norm of $\sqrt{4^{2} + 3^{2}} = \sqrt{16 + 9} = \sqrt{25} = 5$. Give an implementation of a function named norm such that $norm(v, p)$ returns the p-norm value of $v$ and $norm(v)$ returns the Euclidean norm of $v$. You may assume that $v$ is a list of numbers.

\lstinputlisting[title=Exercise C-1.28]{../../GoodrichTamassiaGoldwasser/ch01/c128.py}