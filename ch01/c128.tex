\blogsubsubsection{sssec:ex1_28}{C-1.28}{chap:PythonPrimer}{} \label{sssec:ex1_28}

The p-norm of a vector $v = (v_{1} ,v_{2} ,\ldots,v_{n} )$ in n-dimensional space is defined as:\\

\[
	\left\|v\right\| = \sqrt[p]{v^{p}_{1} + v^{p}_{2} + \ldots + v^{p}_{n}}
\]

For the special case of $p = 2$, this results in the traditional Euclidean norm, which represents the length of the vector. For example, the Euclidean norm of a two-dimensional vector with coordinates $(4,3)$ has a Euclidean norm of $\sqrt{4^{2} + 3^{2}} = \sqrt{16 + 9} = \sqrt{25} = 5$. Give an implementation of a function named norm such that $norm(v, p)$ returns the p-norm value of $v$ and $norm(v)$ returns the Euclidean norm of $v$. You may assume that $v$ is a list of numbers.

\lstinputlisting[title=Exercise C-1.28]{../../GoodrichTamassiaGoldwasser/ch01/c128.py}